\chapter{ETUDE PREALABLE}

Ici il est question pour nous de présenter la planification du travail en relatant d’abord une étude préalable et ensuite la planification proprement dit.

\section{CONTEXTE ET PROBLÉMATIQUE}

Réserver une chambre devrait être la chose la plus facile qui soit mais lorsque on ne sait même pas comment contacter les hôtels ou même de connaitre la position de l’hôtel sur une carte, le fait de réserver deviens une alternative peut priser. Il existe d’autre par des établissements hôtelier offrent des services de qualité a des prix abordables mais qui ne sont pas connu du grand public, Le manque de présence sur internet leur fait passer inaperçu dans le milieu hôtelier.  Dès lors, se pose la question de savoir comment pallier à tous ceci? Nous répondons donc à cette question avec la solution suivante :\\

\begin{center}
 « CONCEPTION ET DÉVELOPPEMENT D’UNE APPLICATION DE RÉSERVATION D’HÔTEL EN LIGNE »
\end{center}

\section{ORGANISATION DU TRAVAIL}
L’organisation de notre travail est calquée de celui d’UML. Nous avons donc reparti notre travail selon les phases suivantes :

\begin{list}{•}{ }
	\item \textbf{L’étude préalable :} c’est la phase au cours de laquelle nous allons collecter des informations sur le fonctionnement de l’application, les analyser, faire des critiques et en fin proposer des solutions.
    \item \textbf{La conception du système :} c’est ici que le gros du travail sera fait. Nous allons effectuer les différents graphiques de l’application et définir son design. Nous allons également concevoir une base de données adéquate à la réalisation de notre application.
    \item \textbf{La production }: c’est la phase de codage coté client et coté serveur. Elle sera effectuée pendant la majorité du temps accordé pour la réalisation du projet. Elle consistera à la création d’une application accessible depuis un smartphone.
    \item \textbf{Test :} c’est la phase où sont testées toutes les fonctionnalités de l’application pour des éventuelles améliorations. Elle sera effectuée à la fin de la phase de production.

\end{list}

\section{DESCRIPTION DE L'EXISTANT}
Des plateformes faisant la réservation en ligne existent et sont disponible sur internet, je vais présenter les majeurs et leurs caractéristiques qui les rendent spéciaux. On a :
\begin{list}{•}{ }
\item Accor Hôtel : plateforme de réservation en ligne pour les hôtels fassent partie du groupe d’hôtels Accor. Accorhotel.com
\item Booking.com : plateforme de réservation d’hôtel en ligne ouverte aux publiques hôtelières ou non
\item Trivago : sa force est de comparer les prix selon vos critères de recherche et de vous retourner le local avec le prix le plus bas.  Il utilise les données d’autres sites (Flux RSS) pour agir efficacement.

\end{list}

\section{CRITIQUE DE L’EXISTANT}
Le problème que tous ces plateforme est commun à tous et c’est le fait qu’aucune n’est vraiment adapter à notre contexte. Vu que les hôtels autre mers suivent rigoureusement une norme qui leurs rend presque similaire dans le système de gestion. Au Cameroun et en Afrique en générale les établissements hôteliers sont divers et de divers culture. D’où le besoin d’une plateforme adapté au contexte qui est le nôtre.

\section{ÉBAUCHE DE SOLUTION}
\subsection{OBJECTIFS GÉNÉRAUX}
Pour des raisons professionnelles et vacances, les gens effectuent dans l’année un nombre important de voyages avec séjour en hôtel, en appart ‘hôtel, en résidence meublée, ou en famille... Parfois pour un aspect pratique, ou une volonté d’optimisation financières et de confort, nous utilisons les services de portails web de groupe hôtelier (accorhaotels.com,…), de portails fédérateurs d’hôtels (hotels.com,…), ou des contacts directs.\\

L’objectif de ce projet, aujourd’hui essentiellement centré sur les hôtels consiste à fédérer au sein d’un site web deux besoins :\\


La volonté des hôteliers : 
\begin{list}{•}{ }
 \item de remplir leurs chambres 
 \item d’offrir une panoplie de services et de découvertes à ses clients tel que :
 \begin{list}{-}{ }
  \item location de voiture 
  \item restaurant 
  \item découvertes et loisirs
 \end{list}  

\end{list}

 La nécessité des clients : 
 
\begin{list}{•}{ }
\item d’avoir un service hébergement en accord avec le descriptif et le tarif, 
\item d’avoir accès à des informations devant facilités leurs différentes activités 
\item de résider dans un cadre sécurisé 
\end{list}

\subsection{CIBLES VISEES}

Première cible : Les résidences hôtelières : hôtels, auberges, … 
\begin{list}{•}{ }
	\item Les résidences hôtelières : hôtels, auberges, … 
    \item Les centres d’intérêts de la cible sont : 
    \begin{list}{•}{ }
     \item L’accroissement de l’activité, 
     \item La relation clientèle, 
     \item L’amélioration des services, 
     \item La connaissance du marché et des tendances 
    \end{list}

\end{list}
\textbf{Deuxième cible :} Les clients 

\begin{list}{•}{ }
 \item Les professionnelles
   \begin{list}{-}{ }
     \item Les PME et PMI 
     \item Les grandes entreprises 
     \item Gouvernement et institutions 
     \item Professionnelles du spectacle et du sport
   \end{list}    
    

  \item Les particuliers 
  \item centres d’intérêts de la cible sont : 
   \begin{list}{-}{ }
   	\item la qualité de services (accueil, séjour, départ, autres services joints) 
    \item le tarif 
    \item le programme de fidélité 
   \end{list}
\end{list}

Troisième cible : les partenaires 
\begin{list}{•}{ }
	\item L’office de tourisme, les communes, les institutions, … \\
Centres d’intérêts de la cible sont : 
  \begin{list}{-}{ }
   \item recherche d’informations 
   \item les données
  \end{list}   
\end{list}

\section{BESOIN FONCTIONNEL}

Les besoins fonctionnels désignent les fonctions que le logiciel doit posséder pour qu’il soit considéré comme fonctionnel. Dans le cas de notre application, elle doit :
\begin{list}{•}{ }
  \item Permettre à un utilisateur de pouvoir rechercher des chambres selon des critères bien spécifiques,
  \item Permettre à un utilisateur de pouvoir réserver une ou plusieurs chambres à travers l’application,
  \item Permettre à un utilisateur (hôteliers) de souscrire les hôtels et le manager.
\end{list}

\section{BESOIN NON FONCTIONNEL}
Les besoins non fonctionnels désignent les conditions à remplir pour qu’un logiciel fonctionne correctement. Dans le cas de notre application, ces conditions doivent être remplir :
\begin{list}{•}{ }
   \item D’un serveur de base de données pour pouvoir stocker les informations concernant l’utilisateur,
   \item D’un serveur web pour pouvoir héberger l’application,
   \item Connexion internet pour avoir accès à la plateforme en ligne.
\end{list}

\section{FONCTIONNALITES}
L’application devra faire ceci :
\begin{enumerate}
  \item \textbf{Partie client :}
   \begin{list}{•}{ }
     \item Authentification d’un utilisateur,
	  \item Consulter la liste des hôtels,
	  \item Consulter les Chambres d’hôtels,
	  \item Rechercher des logements selon leur disponibilité, prix, localité, etc.
	  \item Consulter et éditer son profile.
	  \item Réserver un local.
   \end{list}
  \item \textbf{Partie administrateur(Hôtelier) :}
   \begin{list}{•}{ }
    \item Ajouter son hôtel,
	\item Ajouter des chambres à son hôtel,
	\item Gérer les réservations qui concernent ses locaux,
	\item Ajouter ses offres et accessoire de chambres.
   \end{list}
  \item \textbf{Partie administrateur (Administrateur du site) :}
   \begin{list}{•}{ }
    \item Gérer les demandes de création d’hôtels,
	\item Gérer les utilisateurs et droits,
	\item Répondre aux préoccupations des clients hôteliers.
   \end{list}
\end{enumerate}


\section{CONTRAINTES}
  \subsection{CONTRAINTES MATÉRIELLES}
Pour la réalisation de l’application, nous avons besoin comme matériels :
  \begin{list}{•}{ }
   \item Le serveur d’application doit disposer d’un espace de stockage suffisant et des caractéristiques adéquates.
   \item D’un serveur web APACHE
   \item Le serveur de base de données MySQL.
  \end{list}

  \subsection{DÉLAI}
L’implémentation de la solution choisie doit être réalisée dans un délai de 05q	2q		1q	\\\\\\ mois.

  \subsection{COUT}

Le coût pour réaliser ce projet est de  répartis comme suit :
\begin{list}{•}{ }
 \item Six ordinateurs portables : 1 250 000 Frs,
 \item Une connexion INTERNET : 200,000 Frs,
 \item De la main d'œuvre : 3 000 000Frs à raison de 75000Frs/mois et par personne.
 \item Hébergement Linux chez 1 and 1 : 90,000fr/Année et 55,000 à partir de la deuxième année.
\end{list}















