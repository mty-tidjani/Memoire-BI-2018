\chapter*{Introduction}
\addcontentsline{toc}{chapter}{Introduction}

Les personnes qui se déplacent d’une ville à l’autre dans le monde font tous face aux mêmes obstacles en termes de logement. Il est plutôt rassurent de savoir où on va passer la nuit à l’arrivée d’où l’importance des réservations, pour ça il faudrait néanmoins connaitre des établissements hôtelier de la région. Le cas le plus favorable pour un étranger est lorsqu’il est capable de savoir à quoi exactement s’attendre dans la réservation \(climatisation, lits, repas, etc.\). La réservation en ligne permet d’avoir un aperçu de la chambre, des services, et la position de l’hôtel \(sur une carte\) qui nous accueil donc les mauvaises surprises ne sont pas au rendez-vous. Le Cameroun va accueil la CAN en 2019 et la question est : Comment accueillir dans nos hôtels ces touristes selon leur gouts, portefeuilles et aussi leurs envies de divertissement ?\\

Du 3 Avril 2018 au 31 juillet 2018 j’ais effectue un stage au sein de l’entreprise INSBI \(Institut Business Intelligence\) situé à Logpom-Andem dans la ville de Douala. Au cours de ce stage en tant que développeur, Il m’a été confié la mission de concevoir et réaliser une application de réservation d’hôtel en ligne.\\

Pour mener à bien notre étude, nous avons articulé ce travail autour de trois grandes parties. Dans sa première partie, ce rapport présente une synthèse du contexte professionnel. Elle est suivie d'une partie plus technique dédiée à l’analyse et la conception du system. Dans cette deuxième partie commencerai par présenter l’analyse suivi de la conception du projet, et enfin le développement de l'application. Il y a enfin la troisième partie qui présente le produit obtenue dans sa phase de test.\\


Ce rapport se terminera par une conclusion de ce stage, comment je l'ai ressenti, une explication des différents problèmes rencontres et un bilan personnel.\\
