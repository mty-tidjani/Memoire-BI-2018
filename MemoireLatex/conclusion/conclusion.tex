\chapter*{Conclusion}
\addcontentsline{toc}{chapter}{Conclusion}

Exploiter les données à disposition de l’entreprise afin de leur donner de la valeur ajoutée, tel est le défi des entreprises modernes. Tout au long de notre travail de conception et de réalisation, nous avons essayé de suivre une démarche mixte, alliant de ce fait entre Deux approches connues dans le domaine du Data Warehousing, à savoir l’approche « Besoins d’analyse » et l’approche « Sources de données ». Cette démarche a permis de répondre aux attentes et besoins des utilisateurs tout en exploitant au mieux les données générées par les systèmes opérationnels de manière à anticiper sur des besoins non exprimés. Dans un premier temps nous avons présenté les concepts et méthodes de conception et de réalisation d’un data warehouse. Ensuite nous avons fait un état des lieux sur les systèmes de gestion hôteliers au Cameroun. Chemin faisant nous avons conçu le système qui répond aux questions et aux attentes des décideurs de Hosteline. Les besoins exprimé au départ ont été satisfait et pour avoir quelques coups d’avance nous avons réalisé un système qui répond aussi aux besoins pas encore exprimés. Ainsi à la fin de notre étude nous somme capable de revendiquer des connaissances pratique que nous ne possédions pas au début de ce stage de Cinq mois passé à INSBI. 