\chapter*{Introduction}
\addcontentsline{toc}{chapter}{Introduction}

\paragraph{}
L’avènement des TIC (Technologies de l’information) vient changer notre façon de vivre en la rendent plus simple. Le secteur hôtelier n’est pas exclu de ces changements et proposer ses services sur internet n’est plus un lux mais un besoin fondamentale pour toute entreprise hôtelière qui souhaite rester compétitive. Sur la toile les sites web d’hôtels se multiplient et tout particulier ou entreprise cherchent un ou plusieurs locaux en ligne à une large gamme de site web d’hôtels à sa portée. Le problème quand les possibilités sont large et non exhaustives est que on n’a pas toujours le temps de visiter ces plateformes une par une et comparer leurs offres selon nos préférences. Des plateformes telles que Booking.com, Accordhotel.com et autres offrent une solution qui répond a ce problème de comparaison de prix et d’offres toujours est-il que le contexte et les particularités camerounaises et Africaines n’est pas toujours retrouvée. En effet sur les plateformes de ce type on retrouve des hôtels conventionnels qui respectent certaines normes et standards occidentaux. A titre d’exemple le ministère du tourisme reconnais officiellement 250 hôtels accrédité au Cameroun ce qui ne représente pas 30\% du nombre réel d’établissements hôtelier du pays. Parmi les hébergeurs laissés on retrouve les propriétaires d’auberges, certaines résidences hôtelières, Appartements meublé, Villa et plein d’autre encore. Hosteline s’inspire des plateformes citées plus haut en adaptant le concept au milieu et aux coutumes Camerounaises. En effet Hosteline est une plateforme ouvert à tout type d’hébergeurs et hôteliers qui souhaitent se mettre à la disposition des clients et potentiels clients par le web.
	\paragraph{}
	Lorsque la plateforme sera lancé et tournera à plein régime la quantité de données cumulé sur une année sera énorme et en exploitant la plateforme sur plusieurs année encore ne produira qu’une masse de données conséquente. Pour anticiper sur les besoins de s’informer grâce à ces données  qui iront grandissant, Mr Kendjio le fondateur de Hosteline souhaite se munir d’un système décisionnel pour être prêt à exploiter et tirer avantage de ce volume important de données. La mise en œuvre de ce système décisionnel est le but de la rédaction du présent mémoire. Ce mémoire est organisé en deux parties majeures.\\
Dans la partie une, nous présentons les concepts de base du Business Intelligence dans le chapitre un. Dans le chapitre deux nous parlons des systèmes de gestion hôtelières en générale et de Hosteline en particulier.\\
	Dans la deuxième partie, le chapitre trois donnes les détails sur le cahier des charges. Dans chapitre quatre nous présentons les méthodes d'analyse et de conception de la solution.


