\usepackage[left=2cm,right=2cm,top=2cm,bottom=2cm]{geometry}
\usepackage[utf8]{inputenc}
\usepackage[T1]{fontenc}
\usepackage{amsmath,amsfonts,amssymb}
\usepackage{graphicx}
\usepackage[french]{babel}
\usepackage{listings} % pour insérer du code source
\usepackage{lmodern}
\usepackage{hyperref}
\usepackage{fancyhdr}
\usepackage{graphicx}
\usepackage{eso-pic} % nécessaire pour mettre des images en arrière plan
\usepackage{array}	
\usepackage{fancyhdr} % pour les en-têtes et pieds de pages
\usepackage{fancybox}
\usepackage{hyperref}
\usepackage{layout}
\usepackage{color}
\usepackage{tikz}
\usepackage{pifont}
\usepackage{titlesec}
\usepackage{color}
\usepackage{rotating}
%\usepackage[style=numeric]{biblatex}
\usepackage[Glenn]{fncychap} % pour customiser le style du titre \chapter
\usepackage{type1cm} % aussi pour les lettrines
\usepackage{lettrine}
\usepackage{longtable}
\usetikzlibrary{calc}
\usetikzlibrary{decorations.pathmorphing}
\usepackage{array,multirow,makecell}
\usepackage{setspace}
\usepackage{wrapfig}
\usepackage{sidecap}
\onehalfspacing % pour l'interligne
\setcellgapes{1pt}
\usepackage{array}
\usepackage{minipage-marginpar}
\usepackage{shorttoc} % Pour creer plusieurs tables de matieres
\usepackage{flafter}
\makegapedcells
\newcolumntype{R}[1]{>{\raggedleft\arraybackslash }b{#1}}
\newcolumntype{L}[1]{>{\raggedright\arraybackslash }b{#1}}
\newcolumntype{C}[1]{>{\centering\arraybackslash }b{#1}}
\newcolumntype{M}[1]{>{\centering\arraybackslash}m{#1}}


%customization des \textbullet
\newcommand*{\tikzbullet}[2]{%
  \setbox0=\hbox{\strut}%
  \begin{tikzpicture}
    \useasboundingbox (-.25em,0) rectangle (.25em,\ht0);
    \filldraw[draw=#1,fill=#2] (0,0.5\ht0) circle[radius=.45em];
  \end{tikzpicture}%
}




%%%%%%%%%%%%%%%%%%%style front%%%%%%%%%%%%%%%%%%%%%%%%%%%%%%%%%%%%%%%%% 
\fancypagestyle{front}{%
        \fancyhf{}%on vide l'en-tête
        \fancyfoot[C]{\thepage}
        \renewcommand{\headrulewidth}{0pt}%trait horizontal pour l'en-tête
        \renewcommand{\footrulewidth}{1pt}%trait horizontal pour le pied de page
        }
                
                
                
                
                
                
                
                
%%%%%%%%%%%%%%%%%%%style main%%%%%%%%%%%%%%%%%%%%%%%%%%%%%%%%%%%%


\newcommand{\chaptertoc}[1]{\chapter*{#1}
\markboth{\slshape\MakeUppercase{#1}}{\slshape\MakeUppercase{#1}}}

        \fancypagestyle{main}{%
                \fancyhf{}
                \renewcommand{\chaptermark}[1]{\markboth{\chaptername\ \thechapter.\ ##1}{}} % redéfinition pour avoir ici les titres des chapitres des sections en minuscules
                
                
                \pagestyle{fancy}
                \renewcommand\headrulewidth{1pt}
                \fancyhead[C]{CONCEPTION ET REALISATION D’UN DATA WAREHOUSE POUR LA MISE EN PLACE D’UN SYSTEME DECISIONEL : CAS DE HOSTELINE}
                
                
                \fancyfoot[L]{\textbf{Mémoire rédigé et soutenu par MOMO TESSE Yannick Tidjani}}
                \fancyfoot[R]{\thepage}%
                \renewcommand{\headrulewidth}{1pt}%trait horizontal pour l'en-tête
                \renewcommand{\footrulewidth}{1pt}%trait horizontal pour le pied de page
                }
            
            
            
            
            
            
            
            
            
            
%%%%%%%%%%%%%%%%%%%style pied%%%%%%%%%%%%%%%%%%%%%%%%%%%%%%%%%%%%%%%%%  
        \fancypagestyle{back}{%
                \fancyhf{}%on vide l'en-tête
                \fancyfoot[L]{\textbf{Mémoire rédigé et soutenu par MOMO TESSE Yannick Tidjani}}
                \fancyfoot[R]{\thepage}%
                \renewcommand{\headrulewidth}{0pt}%trait horizontal pour l'en-tête
                \renewcommand{\footrulewidth}{1pt}%trait horizontal pour le pied de pages
                }
                
                
                
                
                
                
%%%%%%%%%%%%%%%%%%% Définition des couleurs %%%%%%%%%%%%%%%%%%%
\definecolor{green1}{HTML}{556627}
\definecolor{green2}{HTML}{B7CA79}
\definecolor{green3}{HTML}{8FCF3C}
\definecolor{green4}{HTML}{1D702D}
\definecolor{green5}{HTML}{456B35}


\definecolor{blue1}{HTML}{1370e9}
\definecolor{blue2}{HTML}{2da9e9}







