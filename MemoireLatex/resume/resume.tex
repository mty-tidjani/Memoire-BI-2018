\chapter{Résumé}


\paragraph{}
	La startup INSBI (Institut Business Intelligence) est une entreprise nouvellement créer qui dans son (Service qu’on propose) a une liste de service qui tourne autour de solutions de digitalisation d’entreprises et Business intelligence. Dans le souci de se faire un nom sur le marché INSBI a produit et lancé son premier produit qui est Hosteline. Hosteline est une plateforme d’hôtellerie en ligne qui a pour cible les propriétaires d’établissements hôteliers  mettent leurs locaux à la disposition des clients qui à leur tour peuvent réserver ces diffèrent locaux. Hosteline est un portail web qui fédère les hôtels et donne un accès comparatif aux clients de cette plateforme. A travers cette plateforme les clients d’hôtels auront des avantages tant sur les prix que sur les facilités que pourront offrir ces établissements hôtelier grâce à son programme de fidélité. Ce système lorsqu’il sera mis en production génèreras une quantité importante de donnée par son système transactionnel. Sachant à quel point ces données sont utiles mais n’informent pas suffisamment a l’état brut, la direction de INSBI souhait mettre en place un système décisionnel pour pouvoir exploiter ces données et en tirer les choix et décisions stratégiques pour faire grandir le produit en offrant une meilleur qualité de service à ses consommateurs. Ces là que nous intervenons afin de mettre en place un système de repporting qui serviras les données issues des systèmes opérationnel dans des formats lisible et facilement interprétable. Notre travail consistait à répondre à un besoin précis consigné dans un cahier avec des contraintes de couts et de délais bien définies. Dans ce mémoire nous mettrons en relief, la méthode de travail, l’évolution des travaux à INSBI et le suivi qu’était le nôtre pendant toute la durée de notre stage. Ce travail nous a permis non seulement de mettre en pratique des enseignements reçus pendant notre formation, d’en apprendre davantage sur les méthodes de fonctionnement des entreprises et aussi sur des notions nouvelles.



